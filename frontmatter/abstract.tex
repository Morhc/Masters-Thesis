\newpage
\TOCadd{Abstract}

\section*{\large Abstract}

O-C shell mergers in massive stars are a site for the production of the p-nuclei via the $\gamma$-process. 
During these mergers, the ingested C-shell material undergoes convective-reactive nucleosynthesis in the O-burning shell where the timescales for advection and nuclear reactions become equal.
1-D stellar models rely on the predictions of mixing length theory which does not match the results of 3-D hydrodynamic simulations in this region.
In this paper, we use the $M_{\mathrm{ZAMS}}=15 M_\odot$, $Z=0.02$ model from the NuGrid stellar data set \cite{ritterNuGridStellarData2018} to create a detailed post-processed model of the O-shell during a merger event to investigate how 3-D macrophysics impacts nucleosynthesis.
This is done by introducing a convective downturn at the bottom of the O-burning shell, varying the rate of ingesting C-shell material, and implementing a dip in the difussion profile due to both a GOSH-like event and partial merger.
In addition to this, we also investigate the impact of varying the input nuclear physics of all photo-disintegration reactions of unstable p-heavy isotopes from Se$-$Po by a factor of $10$ up and down in a Monte Carlo way in various mixing conditions.
The results show that the mixing details have a significant impact on the production of the p-nuclei and influence the impact of the nuclear physics.
Introducing a convective downturn has a non-linear, non-monotonic impact on the production of the p-nuclei, with an average spread of $0.96~\mathrm{dex}$ between MLT and downturn scenarios.
Increasing the C-shell ingestion rate is found to increase production and has a spread in production of $1.22-1.84~\mathrm{dex}$ across MLT and convective downturn scenarios.
GOSH-like and partial merger dips where were found to decrease production in a uniform way by and hasa  spread in production including MLT of $0.51~\mathrm{dex}$.
Finally, the nuclear physics impact was found to have cause a spread in the final mass fraction on average $0.56-0.79~\mathrm{dex}$ across mixing scenarios which is similar to the spread across the various mixing scenarios.
Additionally, we find whether there is a correlation and the strength of it are both dependent on the mixing scenario.
We conclude that understanding the mixing details of the O-C shell merger along with the nuclear physics is critical for understanding the production of p-nuclei and that there are comparable model uncertainties as the nuclear physics.
