\section{Stellar Nucleosynthesis in Massive Stars}

Stellar nucleosynthesis is the process by which elements are formed through nuclear reactions in stars.
\cite{burbidgeSynthesisElementsStars1957} set the foundation for the field describing the synthesis of all elements in hydrostatic and explosive burning.
From theoretical calculations, they described the nucleosynthesis of all isotopes in stars via a series of burning stages and nuclear processes while proposing the likely astrophysical sites.
In the following years, through observational data and computational developments, the work remains a foundational text to stellar nuclear astrophysics.\footnote{One is reminded of Alfred North Whitehead's quote: \textit{The safest general characterization of the European philosophical tradition is that it consists of a series of footnotes to Plato.} Or in our case, B$^2$FH.}
Shortly after, \cite{henyeyNewMethodAutomatic1964} proposed a method for calculating nucleosynthesis of the CNO cycle in stars.
Since then, the field has expanded to modelling the nucleosynthesis of isotopes during all stages of stellar evolution and produce data sets of stellar models at any mass and metallicity \citep{pignatariNuGridStellarData2016, ritterNuGridStellarData2018, battinoNuGridStellarData2019}.

\subsection{Hydrostatic Burning Phases}

Hydrostatic burning describes the nuclear fusion processes that occurs in during the lifetime of a star, where the energy produced by nuclear reactions balance the gravitational forces and allowing the star to maintain hydrostatic equilibrium.
Each of these burning phases is characterized by a specific temperature, density, and timescale where the nucleosynthesis takes place.

\begin{table}[H]
\caption{Details about hydrostatic burning phases in massive stars taken from the $13\text{-}25 M_\odot$ stars in Table 1 of \cite{woosleyEvolutionExplosionMassive2002}}
\begin{tabular}{|l|c|c|c|}
    \hline
    \textbf{Burning Phase} & \textbf{Temperature ($\mathbf{10^7}$ K)} & \textbf{Density (g cm$^{-3}$)} & \textbf{Timescale (yrs)} 
    \\ \hline
    H-burning & 3.44 - 3.81 & 6.66 - 3.81 & 13.5 - 6.70 $\times 10^7$ \\ \hline
    He-burning & 17.2 - 19.6 & 1.73 - 0.762 $\times 10^3$  & 2.67 - 0.839 $\times 10^7$ \\ \hline
    C-burning & 81.5 - 84.1 & 3.13 - 1.29 $\times 10^5$ & 2.82 - 0.522 $\times 10^3$ \\ \hline
    Ne-burning & 169 - 157  & 10.8 - 3.95 $\times 10^6$ & 0.341 - 0.891 \\ \hline
    O-burning & 189 - 209  & 8.19 - 3.60 $\times 10^6$ & 4.77 - 0.402 \\ \hline
    Si-burning &  328 - 365 & 4.83 - 3.01 $\times 10^7$ & 48.8 - 2.0 $\times 10^{-3}$\\
    \hline
\end{tabular}
\end{table}

\subsubsection{H-burning}

H-burning is the first stage of hydrostatic burning in the core of a star. 
There are two processes by which this happens, the pp chain and the CNO cycle.
In low mass stars

The conversion of hydrogen to helium via nuclear fusion occurs primarily through the proton-proton (pp) chains and the CNO cycles. In massive stars, the CNO cycle dominates due to higher central temperatures (\( T \gtrsim 15 \times 10^6\,\mathrm{K} \)):
\begin{equation}
^{12}\mathrm{C}(p,\gamma)^{13}\mathrm{N}(\beta^+)^{13}\mathrm{C}(p,\gamma)^{14}\mathrm{N}(p,\gamma)^{15}\mathrm{O}(\beta^+)^{15}\mathrm{N}(p,\alpha)^{12}\mathrm{C}.
\end{equation}

The net result is four protons converted to a helium nucleus, releasing energy via positrons and neutrinos.

\subsubsection{He-burning}

Once hydrogen is exhausted in the core, helium burning is ignited via the triple-alpha process:
\begin{align}
3\alpha &\rightarrow {}^{12}\mathrm{C}, \
{}^{12}\mathrm{C}(\alpha,\gamma){}^{16}\mathrm{O}.
\end{align}

The $^{12}\mathrm{C}/^{16}\mathrm{O}$ ratio has significant consequences for later burning stages and the final core structure. This phase operates at $T \sim 2\times10^8\,\mathrm{K}$.

\subsubsection{C-burning}

At $T \sim 6\times10^8\,\mathrm{K}$, carbon fusion begins:
\begin{align}
{}^{12}\mathrm{C} + {}^{12}\mathrm{C} &\rightarrow {}^{20}\mathrm{Ne} + \alpha, \
{}^{12}\mathrm{C} + {}^{12}\mathrm{C} &\rightarrow {}^{23}\mathrm{Na} + p, \
{}^{12}\mathrm{C} + {}^{12}\mathrm{C} &\rightarrow {}^{24}\mathrm{Mg} + \gamma.
\end{align}

This burning typically occurs in a convective core or shell, depending on the stellar mass and prior evolution.

\subsubsection{Ne-burning}

Neon burning occurs via photodisintegration reactions at $T \sim 1.2\times10^9\,\mathrm{K}$:
\begin{align}
{}^{20}\mathrm{Ne}(\gamma,\alpha){}^{16}\mathrm{O}, \
{}^{20}\mathrm{Ne}(\alpha,\gamma){}^{24}\mathrm{Mg}.
\end{align}

This process mainly produces Mg and reinforces O production. It is typically confined to a shell.

\subsubsection{O-burning}

At $T \sim 1.5\times10^9\,\mathrm{K}$, oxygen nuclei fuse to form Si-group elements:
\begin{align}
{}^{16}\mathrm{O} + {}^{16}\mathrm{O} &\rightarrow {}^{28}\mathrm{Si} + \alpha, \
{}^{16}\mathrm{O} + {}^{16}\mathrm{O} &\rightarrow {}^{31}\mathrm{P} + p, \
{}^{16}\mathrm{O} + {}^{16}\mathrm{O} &\rightarrow {}^{32}\mathrm{S} + \gamma.
\end{align}

These products mark the onset of the \textquotedblleft alpha-rich\textquotedblright\ freeze-out zone.

\subsubsection{Si-burning}

Silicon burning initiates at \( T \gtrsim 2.7\times10^9\,\mathrm{K} \), leading to nuclear statistical equilibrium (NSE) dominated by iron-peak elements:
\begin{equation}
\text{Si-group} \rightarrow \text{Fe-group} + \text{neutrinos}.
\end{equation}

This final hydrostatic phase proceeds through a series of photo-disintegration and rearrangement reactions that populate nuclei such as $^{54}\mathrm{Fe}$, $^{56}\mathrm{Ni}$, and $^{58}\mathrm{Ni}$.

\subsection{Neutron Capture Processes}

\subsubsection{\texorpdfstring{$s$-Process}{s-Process}}

The slow neutron capture process occurs during He- and C-shell burning via reactions such as:
\begin{equation}
{}^{13}\mathrm{C}(\alpha,n){}^{16}\mathrm{O}, \quad {}^{22}\mathrm{Ne}(\alpha,n){}^{25}\mathrm{Mg}.
\end{equation}

In $12-20 \mathrm{M}_\odot$ stars, the weak s-process dominates, producing elements up to Sr-Y-Zr region.

\subsubsection{\texorpdfstring{$i$-Process}{i-Process}}

The intermediate neutron capture process occurs at neutron densities $n_n \sim 10^{13}\text{–}10^{15}\,\mathrm{cm}^{-3}$, between those typical of the s- and r-processes. It may be triggered in convective-reactive environments, e.g., O-C shell mergers, where hydrogen ingestion drives neutron bursts.

\subsubsection{\texorpdfstring{$r$-Process}{r-Process}}

The rapid neutron capture process requires extreme conditions (\( n_n \gtrsim 10^{20}\,\mathrm{cm}^{-3} \)) and short timescales. While unlikely in $12-20 \mathrm{M}_\odot$ stars, it is included for completeness and is typically associated with neutron star mergers or magneto-rotational CCSNe.

\subsection{\texorpdfstring{$p$-Nuclei Nucleosynthesis}{p-Nuclei Nucleosynthesis}}

The classical p-nuclei are proton-rich isotopes not formed via s- or r-processes. In massive stars, these nuclei are produced via photodisintegration reactions:

\subsubsection{\texorpdfstring{$rp$-Process}{rp-Process}}

Occurs in proton-rich, explosive environments (e.g., X-ray bursts), where rapid proton captures compete with $\beta^+$ decays.

\subsubsection{\texorpdfstring{$\nu p$-Process}{nu-p Process}}

Occurs in the neutrino-driven wind of core-collapse supernovae. Antineutrino absorption on protons produces neutrons that allow for bypassing rp-waiting points:
\begin{equation}
\bar{\nu}_e + p \rightarrow n + e^+.
\end{equation}

\subsubsection{\texorpdfstring{$\gamma$-process}{gamma-process}}

The main process responsible for p-nucleus production in massive stars. Involves successive photodisintegration reactions:
\begin{equation}
(\gamma, n),\ (\gamma, p),\ (\gamma, \alpha).
\end{equation}

\subsubsection{Pre-Explosive \texorpdfstring{$\gamma$-process}{gamma-process}}

Occurs during late hydrostatic O/Ne-shell burning in stars with \(M \gtrsim 12\,\mathrm{M}_\odot\), at $T \sim 2\text{–}3\times10^9\,\mathrm{K}$.

\subsubsection{Explosive \texorpdfstring{$\gamma$-process}{gamma-process}}

Triggered during the passage of the supernova shock through the O-Ne layers, reaching $T \sim 3\text{–}4\times10^9\,\mathrm{K}$.

\subsection{Convective-Reactive Nucleosynthesis}

Convective-reactive nucleosynthesis refers to processes where nuclear burning and convective mixing operate on comparable timescales. This often occurs in unstable mixing regions, such as O-C shell mergers, where hydrogen is ingested into He-, C-, or O-rich layers.

This regime is characterized by the Dämkohler number:
\begin{equation}
D_\alpha = \frac{\tau_{\mathrm{transport}}}{\tau_{\mathrm{reaction}}}.
\end{equation}

\begin{itemize}
\item $D_\alpha \gg 1$: burning is slow; mixing dominates $\Rightarrow$ homogenization.
\item $D_\alpha \ll 1$: burning is fast; negligible mixing $\Rightarrow$ localized nucleosynthesis.
\item $D_\alpha \sim 1$: feedback between burning and transport $\Rightarrow$ fully coupled convective-reactive regime.
\end{itemize}

Such conditions may trigger neutron bursts and exotic nucleosynthesis pathways (e.g., i-process), potentially contributing to chemical anomalies in metal-poor stars and globular clusters.