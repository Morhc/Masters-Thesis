\startfirstchapter{Introduction}
\label{chapter:introduction}

\input chapters/1/1_sec_motivation
\input chapters/1/2_sec_nucleosynthesis
\input chapters/1/3_sec_massive_star
\input chapters/1/4_sec_stellar_modelling
\input chapters/1/5_sec_1d3dmodelling

\section{Thesis Outline}
\subsubsection{Chapter 2: Impact of 3D-macrophysics and nuclear physics on p-nuclei nucleosynthesis in O-C shell mergers}

In Chapter \ref{sec:thepaper}, I describe the impact of a 1-D implementation of 3-D hydrodynamic concerns on the production of the p-nuclei.
I explore how the nature of convective-reactive nucleosynthesis for the p-nuclei, how the mixing concerns mentioned in Section \ref{sec:intro_1dimplementof3d}.
In addition to this, I present the results of a nuclear reaction correlation study and investigate how the mixing concerns change the correlation results.
I also provide a discussion where I explain the potential impacts and limitations of this work.

\subsubsection{Chapter 3: Additional Work}
In Chapter \ref{sec:additional_work}, I describe how this work can continue to be applied to the production of the light odd-Z isotopes.
In particular, I highlight how the isotopes $^{39}\mathrm{K}$, $^{40}\mathrm{K}$, and $^{41}\mathrm{K}$ are impacted by these mixing concerns as it may have relevance to the formation of exoplanets and whether it would have an atmosphere capable of Earth-like life.
I also provide some preliminary work done investigating whether advective-reactive nucleosynthesis matters for the $r$-process in the post-processing of trajectories that have escaped a black hole.