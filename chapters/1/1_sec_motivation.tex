\section{Motivation}

1-D stellar evolution is a field that depends on the assumptions of spherical symmetry and mixing length theory.
While these assumptions have some success describing the early nuclear burning stages of massive stars, they break down in the advanced burning stages \citep{bazanConvectionNucleosynthesisCore1994}.
The convective O-burning shell and its potential merger with a convective C-burning shell are advanced burning stages shortly before the core collapse of a star where 1-D assumptions break down and 3-D hydrodynamics becomes necessary to model the macrophysical flows \citep{meakinTurbulentConvectionStellar2007, jonesIdealizedHydrodynamicSimulations2017, andrassy3DHydrodynamicSimulations2020}.
This poses a problem for nucleosynthesis calculations, as the stellar structure can be complex, with asymmetric hotspots of burning leading to substantial differences in production compared to 1-D \citep{bazanConvectionNucleosynthesisCore1994, rizzutiShellMergersLate2024a}.

The $p$-nuclei are 35 proton-rich stable isotopes heavier than iron that are largely not produced in the neutron capture processes typical of these elements \citep{burbidgeSynthesisElementsStars1957,woosleyPprocessesSupernovae1978}.
Although many astrophysical sites have been proposed for their production \citep{woosleyAlphaProcessRProcess1992, schatzRpprocessNucleosynthesisExtreme1998, rauscherNucleosynthesisMassiveStars2002, travaglioTestingRoleSNe2015}, no single site has been found to have the necessary conditions to produce all of the $p$-nuclei.
One of the sites that has been found to produce the $p$-nuclei is the O-burning shell of a massive star first suggested in the context of a core collapse supernova by \citep{woosleyPprocessesSupernovae1978}. 
Since then, it has been found that pre-explosive O-shell burning is able to produce these nuclei during the O-C shell merger \citep{rauscherNucleosynthesisMassiveStars2002,ritterConvectivereactiveNucleosynthesisSc2018,robertiGprocessNucleosynthesisCorecollapse2023}.
Additionally, it has been found that the nucleosynthesis in the O-C merger dominates the production of the $p$-nuclei regardless of the energy of the supernova explosion \citep{robertiGprocessNucleosynthesisCorecollapse2024b}, highlighting the importance of understanding this astrophysical site.
Additionally, the O-C merger has been found to be a critical site for the production of the odd-Z isotopes P, Cl, K, and Sc which have been underproduced in galactic chemical evolution models \citep{ritterConvectivereactiveNucleosynthesisSc2018, robertiOccurrenceImpactCarbonOxygen2025}.

The O-C shell merger is a 3-D hydrodynamic environment with large asymmetric flows and hotspots of burning that are not captured by 1-D stellar evolution models.
This is problematic for understanding the peculiar nucleosynthesis of this region as current 1-D stellar evolution models rely on mixing length theory to describe the convective motions of this environment.
The purpose of this work is to explore the impact on nucleosynthesis of varying the mixing conditions of the O-shell during C-shell entrainment, particularly for the $p$-nuclei.