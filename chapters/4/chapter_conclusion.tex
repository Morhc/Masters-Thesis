\startchapter{Summary and Conclusion}\label{concl}

\section{Mixing, Nuclear Physics, and the \texorpdfstring{$p$}{p}-Nuclei}\label{sec:mixing_nuclear_pnuclei
}

In this work, I have explored how changing the mixing assumptions made in 1-D stellar evolution models can have a significant impact on the final nucleosynthesis of the $p$-nuclei in O-C shell mergers.
I have shown that the convective downturn, which is a feature of 3-D hydrodynamic simulations, can have a significant impact on the nucleosynthesis of the $p$-nuclei.
The convective downturn leads to a non-monotonic and non-linear impact on the production of the $p$-nuclei, with isotopes of the same element being effected differently, which has direct implications for grains analysis.
I have also shown the impact of the C-shell entrainment rate is also significant.
While most species are uniformly boosted by the entrainment, the large variation depending on the true entrainment rate means that there exists large uncertainties there.
Potential convective quenching in the profile either GOSH-like at peak Ne-burning or at the top of the shell due to hydrodynamic feedback only partially merging the O- and C-shells decrease the production of the $p$-nuclei also was shown.
Finally, I have shown that while the impact from varying the input nuclear physics has a comparable impact to any individual mixing scenario, different scenearios result in different nuclear physics impacts.
Addtionally, whether a species and reaction rate are correlated is largely dependent on the mixing scenario with few shared correlated rates.
This demonstrates the importance of understanding the mixing conditions in the O-C shell merger environment to accurately model the nucleosynthesis of the $p$-nuclei.

\section{Light Odd-Z Isotope Production}\label{sec:lightoddZ}

Although the work is preliminary, current results show that the light odd-Z elements are also significantly impacted by the 1-D mixing asssumptions showing all the same features as the $p$-nuclei.
This shows the importance of understanding the mixing conditions in O-C shell mergers for O-C merger nucleosynthesis generically, and not just the more obscure $p$-nuclei.
The convective-reactive environment is a complex scenario for nucleosynthesis, and this emphasizes the importance of understanding and connecting 3-D hydrodynamic simulations to nucleosynthesis.

